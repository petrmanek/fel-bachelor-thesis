\chapter{Conclusion}

\section{Future of the Application}
At the time of writing this work, the server application is hosted at IEAP. The web visualization UI is publicly accessible online\footnote{\url{http://atlastpx.utef.cvut.cz}}, periodically updated with the latest ATLAS-TPX footage. Its internal components consitute a rudimentary data warehouse, which not only offers several terabytes of research data, but also serves to further promote scientific works done by IEAP, CTU, ATLAS and Medipix collaboration at CERN.

In the future, the physical machine hosting the application is expected to be upgraded and relocated to CERN. Consequently, the JSTP server would be updated for compatibility with EOS, which would be used to substitute a conventional file system. The PostgreSQL server used to manage the index database could be also replaced by CERN's OnDemandDb service. Moreover, while hosted at CERN, the JSTP server may be modified to directly receive data streams of captured information in the multi-frame format and process them automatically in a queue. This setup might reduce the time interval between data acquisition and visualization from days to hours or possibly even minutes, with respect to the limitations of hardware and local area network available at CERN.

A slightly modified version of the server application might also be utilized to offer access to footage captured by TPX detectors installed in different experiments operated in collaboration with IEAP. At the present time, the author of this work is investigating its possible deployment in the MoEDAL experiment\footnote{Similarly to ATLAS-TPX, MoEDAL-TPX is a network of Timepix devices installed within the MoEDAL experiment at CERN.} CERN and the RISESat satellite collaboration\footnote{RISESat is a microsatellite mission carrying several scientific instruments including a Timepix detector.}. In the end, it is his belief that this application has the potential to become an data visualization software for any scientific experiment gathering data from TPX detectors, capable of efficiently operating with big amounts of research data.
