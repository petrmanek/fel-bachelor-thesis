\chapter{Conclusion}

In the initial state, there was no sophisticated visualization for data from the ATLAS-TPX network. This work formalized a universal database, capable of holding footage from 15 detectors in the ROOT file format. It also defined an auxiliary database built atop SQL, which allowed the acceleration of the most common queries by the application of multiple layers of data indexing. An interactive web visualization UI was designed and implemented along with a server-side component to transcode TPX data into JSTP, a proprietary data format defined to efficiently encode frames for network transfers. The architecture of the server implementation allows multiple users to operate the application simultanously by the use of multi-threading.

The web visualization UI is currently running and available online\footnote{\url{http://atlastpx.utef.cvut.cz}}. Various performance optimizations described in this work have significantly decreased access latency and reduced hardware requirements on the client-side rendering process.
