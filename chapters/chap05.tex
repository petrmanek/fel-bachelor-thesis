\chapter{Conclusion}

\section{Data Import}
Since the JSTP server uses the index database to locate data files based on any given start time, all data files subject to visualization must be stored in the ROOT format and referenced in the \texttt{rootfiles} SQL table.

Accounting for the ever-growing nature of ATLAS-TPX footage, a periodical prodcedure needs to be performed every time new data arrives from CERN, in order to keep the visualization UI up-to-date. At the time of writing this work, this procedure is semi-automated and initiated manually every day by the researchers at IEAP.

\subsection{Processing Stages}
At CERN, the control PC generates raw data files from the read-out interface of every detector in the network. Data is produced on a hourly basis and saved in the form of multi-frame files. Using FTP, these files are transferred into a temporary directory on the target hard drive. When all file transfers are completed and the validity of files is confirmed, several scripts designed to check consistency of detector acquisition are executed. These scripts analyze contents of received files, reading common configuration values such as the acquisition time or bias voltage, and attempt to find variations between individual frames.

Should these scripts succeed in detecting suspicious values, the files in question are moved into a separate directory and await further inspection by researchers. Otherwise, they move on to the next stage of processing. In this stage, captured frames are subjected to the cluster analysis (for more information about this process, see section \ref{intro:cluster-analysis}). Should the frames be captured in the TOT mode, at this point calibration data is used in combination with the method described in \cite{Jakubek2011S262} to calculate energy values. At the end of this process, ROOT files are produced.

Since the original multi-frame files are not needed anymore, they can be discarded without data loss (or compressed for the purposes of long-term archiving). The ROOT files are moved from the temporary directory to their final location on the hard drive containing the ATLAS-TPX footage database. Subsequently, a dedicated instance of the JSTP data server application is configured to generate information in the index database, in effect registering them for retrieval of JSTP information. From this point onward, the JSTP server as well as the web visualization UI are able to read frames stored within newly added files.

\subsection{Automation of the Procedure}
\label{import-automation}
At the present time, the procedure of extending TPX database with new footage is time-consuming as its processing stages often perform similar operations repeatedly for diferent purposes. For example, a single enumeration of multi-frame files should suffice both for consistency checking and cluster analysis. Furthermore, index data corresponding with produced ROOT files can be generated at the time of their production.

It is however worth noting that this procedure was originally designed in the fall of 2015 with the sole purpose of transferring data from CERN to IEAP. Its automation is currently under investigation.

\section{Future of the Application}
At the time of writing this work, the server application is hosted at IEAP. The web visualization UI is publicly accessible online\footnote{\url{http://atlastpx.utef.cvut.cz}}, periodically updated with the latest ATLAS-TPX footage. Its internal components consitute a rudimentary data warehouse, which not only offers several terabytes of research data, but also serves to further promote scientific works done by IEAP, CTU, ATLAS and Medipix collaboration at CERN.

In the future, the physical machine hosting the application is expected to be upgraded and relocated to CERN. Consequently, the JSTP server would be updated for compatibility with EOS, which would be used to substitute a conventional file system. The PostgreSQL server used to manage the index database could be also replaced by CERN's OnDemandDb service. Moreover, while hosted at CERN, the JSTP server may be modified to directly receive data streams of captured information in the multi-frame format and process them automatically in a queue. This setup might reduce the time interval between data acquisition and visualization from days to hours or possibly even minutes, with respect to the limitations of hardware and local area network available at CERN.

A slightly modified version of the server application might also be utilized to offer access to footage captured by TPX detectors installed in different experiments operated in collaboration with IEAP. At the present time, the author of this work is investigating its possible deployment in the MoEDAL experiment CERN and the RISESat satellite collaboration. In the end, it is his belief that this application has the potential to become an data visualization software for any scientific experiment gathering data from TPX detectors, capable of efficiently operating with big amounts of research data.
