\chapter{Communication Protocol}
% 3. Komunikační protokol pro přenos dat

In this chapter, we move our focus from the data itself to a communication protocol we use to transmit the data for the purposes of visualization. We describe the overall scheme of communication and define the protocol in a formal way.

\section{Client-Server Model}
Let us first state some assumptions about our system. We expect to have multiple users connecting to our data storage over a local area network or through the Internet. We assume that our users will want to see and possibly further examine some of the frames captured by the ATLAS-TPX network.

We do not expect to transmit all information from our data files, nor do we want to send continuous footage at streaming speeds. Instead, we define a protocol to enable simple access to detector data. The goal is to provide brief overview of recent detector operation with emphasis on any irregular or pattern-defying events.

\subsection{Considerations}
It is worth noting that in designing our system, we would like to uphold a multi-layered architecture. This way, there will be strict distinctions between individual layers of the system (and the tasks they perform), making them in effect easily extendable, substitutable and perhaps even portable to other applications.

Other benefit of this approach is that users of our system will always have freedom to choose a layer, with which they wish to interact. We may imagine this as follows. Users, who want a quick peek at the detector operation without any effort, may use the web visualization UI. Users, who want to retrieve data for experimentation or statistical aggregation, might utilize SQL or the communication protocol we define in this very chapter. Lastly, users in need of data enumeration or manipulation can connect to the database storage device and directly download data files using some of the supported network transfer protocols.

% CITACE: konkurenční projekty
We should also consider extensibility of our protocol. With multiple concurrent projects such as MoEDAL-TPX\footnote{Similarly to ATLAS-TPX, MoEDAL-TPX is a network of Timepix device installed within the MoEDAL experiment at CERN.}, SATRAM\footnote{SATRAM is a technology demonstration device carrying Timepix position-sensitive semiconductor pixel detector on board ESA’s Proba V satellite. \url{http://satram.utef.cvut.cz/}} and RISESat\footnote{RISESat is a microsatellite mission carrying several scientific instruments including a Timepix detector.}, it is likely that the same protocol will be used for compatibility reasons in other applications as well. Our protocol should therefore allow for limited variability, gracefully handling minor alterations in transmitted data structures.

\begin{figure}[t]
\begin{center}
	\begin{tikzpicture}[node distance=3pt,
	blueb/.style={
	  draw=black,
	  rounded corners,
	  text width=2.5cm,
	  font=\scriptsize,
	  align=center,
	  text height=12pt,
	  text depth=9pt
	},
	layerb/.style={
	  blueb,
	  draw=none
	},
	proprietaryb/.style={
	  blueb,
	  fill=gray!30
	}]

	\node[layerb] (Clients) {\textbf{Applications}};
	\node[proprietaryb,right=of Clients,text width=5cm+10pt] (VizUI) {Web visualization UI};
	\node[blueb,right=of VizUI,text width=5cm+10pt] (Apps) {Others \dots};

	\node[layerb,below=of Clients] (Protocols) {\textbf{Protocols}};
	\node[proprietaryb,right=of Protocols] (RTAP) {RTAP};
	\node[blueb,right=of RTAP] (HTTP) {HTTP};
	\node[blueb,right=of HTTP] (FileP) {FTP, SMB};
	\node[blueb,right=of FileP] (SQLP) {SQL queries};

	\node[layerb,below=of Protocols] (Servers) {\textbf{Servers}};
	\node[proprietaryb,right=of Servers] (DataS) {Data RPC};
	\node[blueb,right=of DataS] (WebS) {Static Web};
	\node[blueb,right=of WebS] (FileS) {File Servers};
	\node[blueb,right=of FileS] (SQLS) {SQL};

	\node[layerb,below=of Servers] (Data) {\textbf{Data Store}};
	\node[blueb,right=of Data] (ROOT) {ROOT};
	\node[blueb,right=of ROOT,text width=5cm+10pt] (MF) {Multi-frame \& single-frame};
	\node[proprietaryb,right=of MF] (SQLD) {Index DB};

	\node[blueb,below=2.4cm of HTTP,text width=13cm+26pt] (FileSystem) {UNIX-like file system (possibly EOS)};

	\begin{pgfonlayer}{background}
	\draw[blueb,draw=black] 
	  ([xshift=-8pt,yshift=8pt]current bounding box.north west) rectangle 
	  ([xshift=8pt,yshift=-8pt]current bounding box.south east);
	\end{pgfonlayer}
	\end{tikzpicture}

\caption{A multi-layered system. Proprietary components are emphasized by gray color.}
\label{fig:multilayered-diagram}
\end{center}
\end{figure}

\section{Requirements}
%  - Definice požadavků na datový server, možnosti pro trvalé úložiště dat.

\section{Underlying Standards}
%  - Návrh technického formátu protokolu.

\section{Web Methods}
%  - Stanovení metod, které bude server obsluhovat.
%  - Stručná technická dokumentace navrženého protokolu.
