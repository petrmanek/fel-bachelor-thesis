\chapter{Documentation of JSTP Web~Methods}
\label{apx:jstp-doc}
This chapter includes detailed documentation of the JSTP web service along with protocol conventions, parameter descriptions and examples of requests and responses.

\section{Conventions}
Note that when referring to the service endpoint in method URLs, we use \texttt{<endpoint>} as a stand-in string.
TODO

\section{Detector List}
\label{apx:jstp-sensors}
To execute this method, a client must initiate GET request to \texttt{<endpoint>/sensors} without any parameters. When successful, the server responds by returning an array of objects, each of which corresponds to a single device in the network. Example of such response is provided in Listing \ref{lst:jstp-sensors}. Every object in the array is guaranteed to contain:

\begin{description}
	\item[\texttt{sid}]
	Unique numeric identifier of the device retrieved from the index database.

	\item[\texttt{name}]
	Readable name of the device.
\end{description}

\begin{listing}
	\inputjson{code/jstp-sensors.json}
    \caption{Example response containing a list of two devices.}
    \label{lst:jstp-sensors}
\end{listing}


\section{Overview of Acquisition}
\label{apx:jstp-timeline}
To execute this method, a client must initiate POST request to \texttt{<endpoint>/timeline}. The request body must contain a JSON object with \textit{all} parameter values. You can examine an example request in Listing \ref{lst:jstp-timeline-request}.

\begin{listing}
	\inputjson{code/jstp-timeline-request.json}
    \caption{Example request body with time period starting at July 28, 2015 at 3:00 AM and ending at 6:00 AM. Data from 2 detectors is requested to be normalized and grouped by every hour. Response is expected to contain exactly 3 intervals.}
    \label{lst:jstp-timeline-request}
\end{listing}

When successful, the server responds by returning an array of objects, each of which responds to a single interval in the time period. For example response, see Listing \ref{lst:jstp-timeline-response}. Every object in the array is guaranteed to contain:

\begin{description}
	\item[\texttt{time}]
	UNIX timestamp in UTC of the start time of the interval. End time of the interval can be calculated at by adding \texttt{groupPeriod} to this value.

	\item[\texttt{frames}]
	Number of frames aggregated in the time interval.
	
	\item[\texttt{occupancy}]
	Count of non-zero pixels in all aggregated frames, indicating the levels of saturation. The maximum possible occupancy is equal to the product of pixels in a single sensor layers, the number of sensor layers and the number of aggregated frames in the interval.
	
	\item[\texttt{counts}]
	Array of counts of clusters in all aggregated frames, differentiated by their type classification. Counts are provided in the order: dots, small blobs, heavy blobs, heavy tracks, straight tracks, curly tracks.

	If the calculations are normalized, individual contributions to these counts from every frame are divided by frame's acquisition time, yielding overall flux instead of counts.
\end{description}

\begin{listing}
	\inputjson{code/jstp-timeline-response.json}
    \caption{Example response to the request from Listing \ref{lst:jstp-timeline-request}.}
    \label{lst:jstp-timeline-response}
\end{listing}

\section{Frame Search}
\label{apx:jstp-frame}
To execute this method, a client must initiate POST request to \texttt{<endpoint>/frame}. The request body must contain a JSON object with \textit{all} parameter values:

\begin{description}
	\item[\texttt{time}]
	UNIX timestamp in UTC of the search time parameter.

	\item[\texttt{sensors}]
	Array of distinct \texttt{sid} values of the devices, from which we wish to retrieve data. This array must not be empty.

	\item[\texttt{searchMode}]
	A non-negative integer value specifying the algorithm to be used in the search operation. Possible values are 0 for the Sequential Forward Mode and 1 for the Sequential Backward Mode.

	\item[\texttt{integralFrames}]
	A positive integer not greater than 100 controlling the number of frames integrated in time. Value equal to 1 retrieves only a single frame per device.
\end{description}

\begin{listing}
	\inputjson{code/jstp-frames-request.json} % TODO
    \caption{Example request body with time parameter equal to July 28, 2015, 3:00 AM. A single frame captured by a single detector is requested to be located by the Sequential Forward Mode.}
    \label{lst:jstp-frames-request}
\end{listing}

For an example request, see Listing \ref{lst:jstp-frames-request}. In response, the server returns an object containing \texttt{foundTime}, the start time of the master frame, and \texttt{frames}, an array of objects corresponding with frames captured by  every device in order, in which they were referenced in the \texttt{sensors} array. Every object is guaranteed to contain:

\begin{description}
	\item[\texttt{rootFile}]
	Path to the ROOT file, from which this frame was extracted (in the server's file system).

	\item[\texttt{rootFrameIndex}]
	Index of the entry in ROOT file's \texttt{dscData} tree, containing information about detector configuration.

	\item[\texttt{rootFirstClusterIndex}]
	Index of the first entry in ROOT file's \texttt{clusterFile} tree, corresponding with the first cluster in the frame. If no such entry exists, this value is null or negative.

	\item[\texttt{layers}]
	Number of detector's sensor layers.

	\item[\texttt{startTime}]
	UNIX timestamp in UTC of the start time of acquisition.

	\item[\texttt{acquisitionTime}]
	The acquisition time (the length of acquisition) in seconds.

	\item[\texttt{biasVoltage}]
	TODO

	\item[\texttt{mode}]
	TODO

	\item[\texttt{chipboardId}]
	TODO

	\item[\texttt{maskedPixels}]
	TODO

	\item[\texttt{calibrationConstants}]
	TODO (only TOT)

	\item[\texttt{clusters}]
	TODO
\end{description}
